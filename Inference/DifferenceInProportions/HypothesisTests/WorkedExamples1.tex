
\documentclass[a4paper,12pt]{article}
%%%%%%%%%%%%%%%%%%%%%%%%%%%%%%%%%%%%%%%%%%%%%%%%%%%%%%%%%%%%%%%%%%%%%%%%%%%%%%%%%%%%%%%%%%%%%%%%%%%%%%%%%%%%%%%%%%%%%%%%%%%%%%%%%%%%%%%%%%%%%%%%%%%%%%%%%%%%%%%%%%%%%%%%%%%%%%%%%%%%%%%%%%%%%%%%%%%%%%%%%%%%%%%%%%%%%%%%%%%%%%%%%%%%%%%%%%%%%%%%%%%%%%%%%%%%
\usepackage{eurosym}
\usepackage{vmargin}
\usepackage{amsmath}
\usepackage{graphics}
\usepackage{epsfig}
\usepackage{subfigure}
\usepackage{enumerate}
\usepackage{fancyhdr}
\usepackage{framed}

\setcounter{MaxMatrixCols}{10}
%TCIDATA{OutputFilter=LATEX.DLL}
%TCIDATA{Version=5.00.0.2570}
%TCIDATA{<META NAME="SaveForMode"CONTENT="1">}
%TCIDATA{LastRevised=Wednesday, February 23, 201113:24:34}
%TCIDATA{<META NAME="GraphicsSave" CONTENT="32">}
%TCIDATA{Language=American English}

\pagestyle{fancy}
\setmarginsrb{20mm}{0mm}{20mm}{25mm}{12mm}{11mm}{0mm}{11mm}
\lhead{StatsResource} \rhead{Worked Examples} \chead{Inference Procedures} %\input{tcilatex}

\begin{document}
\large 
\noindent  A study was made of children who were hospitalized as a result of a car accident. 280 of the children were not wearing seat belts and 98 of these were seriously injured. 130 children wore seat belts and 26 were seriously injured. 
\bigskip
\begin{enumerate}[(a)]
\item Test the hypothesis that the rate of serious injury is the same for children who wear a seat belt or not. Clearly state your null and alternative hypotheses and your conclusion. Use a significance level of 5\%.
\end{enumerate}
\bigskip

\begin{framed}
		\noindent \textbf{Hypothesis Testing : Steps when using Critical Value approach}
		
		\begin{itemize}
			\item[1] Formally State the Null and Alternative Hypothesis \smallskip
			{
				\begin{itemize}
					\item[$\ast$] \textbf{ALWAYS} include a short written description of both hypotheses.
					\item[$\ast$] State hypotheses in mathematical notation where possible.
					
				\end{itemize}
			}
			\item[2] Calculate Test Statistic (TS) using relevant formulas.\smallskip
			\item[3] Determine the Critical Value (CV) from tables. \smallskip
			\item[4] By comparing the values of the Test Statistic and Critical Value, decide whether to reject the Null Hypothesis.
		\end{itemize}
	\end{framed}

%%%%%%%%%%%%%%%%%%%%%%%%%%%%%%%%%%%%%%%%%%%%%%%%%%%%%%%
\newpage 
\subsection*{Statement of Hypotheses}

\begin{itemize}
    \item The Null Hypothesis ($H_0$) proposes that there is no difference in injury rates if you wear a seatbelt or not.
    \item The Alternative Hypothesis ($H_1$) proposes that there is a difference in injury rates if you wear a seatbelt or not.
\end{itemize}
\medskip
\begin{description}
\item[$H_0$:] $\pi_1 = \pi_2$  - i.e. Population proportions are equal to each other.
\item[$H_1$:] $\pi_1 \neq \pi_2$ - i.e. Population proportions are different to each other.
\end{description}
\medskip 
\noindent Equivalently, the hypotheses can be expressed in terms of difference in proportions.
\begin{description}
\item[$H_0$:] $\pi_1 - \pi_2 = 0$ - i.e. Difference in population proportions is zero.
\item[$H_1$:] $\pi_1 - \pi_2 \neq 0$ - i.e. Difference in population proportions is not zero.
\end{description}

%%%%%%%%%%%%%%%%%%%%%%%%%%%%%%%%%%%%%%%%%%%%%%%%%%%%%%%
\newpage
\subsection*{Computing the Test Statistic}
\noindent \textbf{Point Estimate}

\begin{itemize}
    \item The sample proportions for both groups are
    \[ \hat{p}_1 = \frac{98}{280} = 0.35 \;\;(\mbox{i.e.}\; 35\%) \] 
    \[ \hat{p}_2 = \frac{26}{130} = 0.20 \;\;(\mbox{i.e.}\; 20\%) \]  
    \item The point estimate is the difference between these proportions:
    \[ \hat{p}_1  - \hat{p}_2  = 0.15 \;\;(\mbox{i.e.}\; 15\%) \]
\end{itemize}
\noindent \textbf{Standard Errors}
\begin{itemize}
    \item The Standard Error Formula are typically provided in the exam paper, or an accompanying document.
    \item Each type of hypothesis test has a different Standard Error Formula.
\end{itemize}
\bigskip
\begin{framed}
\noindent \textbf{Standard Error for Difference in Proportions}\\
\noindent {Two large independent samples}
\begin{eqnarray*}
	S.E.(\hat{P_1}-\hat{P_2})&=&\sqrt{ \bar{p}\times(1-\bar{p}) \times \left(\frac{1}{n_1}+\frac{1}{n_2}\right)}.\\
\end{eqnarray*}
\begin{itemize}
    \item $\bar{p}$ is the overall proportion
\[    \bar{p} = \frac{x_1 + x_2}{n_1+n_2}\]
\end{itemize}
\end{framed}
\[    \bar{p} = \frac{x_1 + x_2}{n_1+n_2} = \frac{98 + 26}{280 + 130} = \frac{124}{410} = 0.3025 \]

\noindent Using percentages, the aggregate proportion is 30.25\%.

\begin{eqnarray*}
	S.E.(\hat{P_1}-\hat{P_2})&=&\sqrt{ 30.25 \times 69.75 \times \left(\frac{1}{280}+\frac{1}{130}\right)}.\\
	&=&\sqrt{ 2109.938 \times 0.01126}.\\
	&=&\sqrt{23.7579}\\
	&=& 4.8742 \\
\end{eqnarray*}


\noindent \textbf{General Structure of the Test Statistic}
\begin{framed}
\[ TS = {\mbox{Point Estimate} - \mbox{Expected Parameter Value under $H_{0}$} \over \mbox{Std. Error}}\]
\end{framed}
\medskip

\begin{eqnarray*}
TS &=& \frac{ 15\;-\; 0}{4.8742 } \\
& & \\
&=& 3.077\\
\end{eqnarray*}
%%%%%%%%%%%%%%%%%%%%%%%%%%%%%%%%%%%%%%%%%%%%%%%%%%%%%%%
\newpage 
\subsection*{Determine the Critical Value}

\begin{itemize}
\item The critical value is determined from Statistical Tables.
\item Determination depends on the significance level $\alpha$, the sample size $n$ and the number of tails in the test (i.e. is it a one-tailed or two tailed test?)
\end{itemize}

\noindent \textbf{Significance Level}

\begin{itemize}
\item In hypothesis testing, the significance level $\alpha$ is the criterion used for rejecting the null hypothesis. 
%\item The significance level of a statistical hypothesis test is a fixed probability of wrongly rejecting the null hypothesis $H_0$, if it is in fact true.
%\item Equivalently, the significance level (denoted by $\alpha$) is the probability that the test statistics will fall into the \textbf{\emph{critical region}}, when the null hypothesis is actually true. 

\item In University courses, common choices for $\alpha$ are $0.05$ and $0.01$. We will use $\alpha =0.05$ (i.e. 5\%).
\item We have a large aggregate sample size ($n_1 + n_2 > 30)$.
\item The hypothesis test is 2-tailed procedure.
\item The Critical Value is 1.96.
\end{itemize}


%%%%%%%%%%%%%%%%%%%%%%%%%%%%%%%%%%%%%%%%%%%%%%%%%%%%%%%
\newpage 
\subsection*{Decision}



\begin{framed}
\noindent \textbf{What are the conclusions}
	\begin{itemize} 
		
		\item[Yes:] We \textbf{reject the Null} Hypothesis. \\ \textit{We have sufficient evidence against Null Hypothesis.}
		
		\item[No:] We \textbf{fail to reject} Null hypothesis. \\ \textit{We do not have sufficient evident against Null Hypothesis.}
	\end{itemize}	
	{	\normalsize
		N.B. Note the terminology that we are using. Also note exactly what our conclusion is: We are talking about strength of evidence, rather than what is true or false.}
	
\end{framed}
	\smallskip
\begin{framed}
\begin{itemize}
	\item 	If $TS > CV$ , where $CV$ is a critical value corresponding to the sample size and confidence level, then reject the null hypothesis. 
	\item  If $TS \leq CV$ , we fail to reject. null hypothesis. i.e. Not enough evidence. 
	\end{itemize}	
\end{framed}	

\noindent Here $TS > CV$ , i.e. ($3.08 > 1.96$), so we reject the null hypothesis. i.e. evidence of a significant difference in proportions. Wearing a seatbelt plays a significant difference in injury rates.
%%%%%%%%%%%%%%%%%%%%%%%%%%
\newpage
BLANK
		
	
\end{document}	
