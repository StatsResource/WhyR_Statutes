\documentclass[a4paper,12pt]{article}
%%%%%%%%%%%%%%%%%%%%%%%%%%%%%%%%%%%%%%%%%%%%%%%%%%%%%%%%%%%%%%%%%%%%%%%%%%%%%%%%%%%%%%%%%%%%%%%%%%%%%%%%%%%%%%%%%%%%%%%%%%%%%%%%%%%%%%%%%%%%%%%%%%%%%%%%%%%%%%%%%%%%%%%%%%%%%%%%%%%%%%%%%%%%%%%%%%%%%%%%%%%%%%%%%%%%%%%%%%%%%%%%%%%%%%%%%%%%%%%%%%%%%%%%%%%%
\usepackage{eurosym}
\usepackage{vmargin}
\usepackage{amsmath}
\usepackage{graphics}
\usepackage{epsfig}
\usepackage{subfigure}
\usepackage{enumerate}
\usepackage{fancyhdr}

\setcounter{MaxMatrixCols}{10}
%TCIDATA{OutputFilter=LATEX.DLL}
%TCIDATA{Version=5.00.0.2570}
%TCIDATA{<META NAME="SaveForMode"CONTENT="1">}
%TCIDATA{LastRevised=Wednesday, February 23, 201113:24:34}
%TCIDATA{<META NAME="GraphicsSave" CONTENT="32">}
%TCIDATA{Language=American English}

\pagestyle{fancy}
\setmarginsrb{20mm}{0mm}{20mm}{25mm}{12mm}{11mm}{0mm}{11mm}
\lhead{StatsResource} \rhead{Worked Example} \chead{Discrete Random Variables} %\input{tcilatex}

\begin{document}
\Large 
\section*{Discrete Random Variables: Worked Example}

Suppose $X$ is a random variable with 
\begin{itemize}
\item $E(X^2)=3.6$
\item $P(X=2)=0.6$
\item $P(X=3)=0.1$
\end{itemize}


\noindent \textbf{Questions:}
\begin{itemize}
\item[(a)] The random variable takes just one other value besides 2 and 3. This value is greater than 0. What is this value?
\item[(b)] Calculate the expected value of $X$.
\item[(c)] What is the variance of $X$?
\end{itemize}

%-------------------------------------------------------------%




\noindent \textbf{Part a}
\begin{itemize}
\item Firstly we have to determine the missing value (let's call it $k$), and then determine the probability of that value. 
\item We know that $E(X^2)=3.6$. Let use the approach for computing $E(X^2)$.

\end{itemize}

\begin{center}
\begin{tabular}{|c|c|c|c|}
\hline
$x_i$ & \phantom{sp}2\phantom{sp} & \phantom{sp}3\phantom{sp} & \phantom{sp}k\phantom{sp} \\ \hline
$x^2_i$ & 4 & 9 & $k^2$ \\ \hline
$P(x_i)$ & 0.6 &  0.1 &  \\ \hline 
\end{tabular}
\end{center}


\begin{center}
\begin{tabular}{|c|c|c|c|}
\hline
$x_i$ & \phantom{sp}2\phantom{sp} & \phantom{sp}3\phantom{sp} & \phantom{sp}k\phantom{sp} \\ \hline
$x^2_i$ & 4 & 9 & $k^2$ \\ \hline
$P(x_i)$ & 0.6 &  0.1 & \textbf{0.3} \\ \hline 
\end{tabular}
\end{center}







\begin{center}
\begin{tabular}{|c|c|c|c|}
\hline
$x_i$ & \phantom{sp}2\phantom{sp} & \phantom{sp}3\phantom{sp} & \phantom{sp}k\phantom{sp} \\ \hline
$x^2_i$ & 4 & 9 & $k^2$ \\ \hline
$p(x_i)$ & 0.6 &  0.1 & 0.3 \\ \hline 
\end{tabular}
\end{center}


\[ E(X^2) =  \sum  x^2_i \cdot p(x_i)  = 3.6 \]

\[(4 \times 0.6) + (9\times 0.1) + (k^2\times 0.3) = 3.6 \]







\begin{center}
\begin{tabular}{|c|c|c|c|}
\hline
$x_i$ & \phantom{sp}2\phantom{sp} & \phantom{sp}3\phantom{sp} & \phantom{sp}k\phantom{sp} \\ \hline
$x^2_i$ & 4 & 9 & $k^2$ \\ \hline
$p(x_i)$ & 0.6 &  0.1 & 0.3 \\ \hline 
\end{tabular}
\end{center}


\begin{itemize}
\item $ E(X^2) =  \sum  x^2_i \cdot p(x_i)  = 3.6 $
\item $(4 \times 0.6) + (9\times 0.1) + (k^2\times 0.3) = 3.6 $
\item $2.4 + 0.9 + (k^2\times 0.3) = 3.6 $

\item $ 3.3 + 0.3k^2 = 3.6$

\item $0.3k^2 =0.3$
\end{itemize}


\begin{center}
$k^2 = 1$  \qquad Therefore $k=1$
\end{center}

%-------------------------------------------------------------%



\noindent \textbf{Part b}\\
Compute the variance of $X$

\[ \mbox{Var}(x) = E(X^2) - \{E(X)\}^2 \]


\begin{itemize}
\item We already know $E(X^2) =3.6$
\item Need to compute $E(X)$.
\end{itemize}

%-------------------------------------------------------------%



\begin{center}
\begin{tabular}{|c|c|c|c|}
\hline
$x_i$ & 2 & 3 & 1 \\ \hline 
$p(x_i)$ & 0.6 &  0.1 & 0.3 \\ \hline
\end{tabular}
\end{center}

\[ E(X) =  \sum  x_i \cdot p(x_i)   \]










\[E(X) = (2\times 0.6) + (3 \times 0.1) + (1 \times 0.3) = 1.8 \]



%-------------------------------------------------------------%





\noindent \textbf{Part c}\\
Compute the variance of $X$

\[ \mbox{Var}(X) = E(X^2) - \{E(X)\}^2 \]

\[ \mbox{Var}(X) = 3.6 - \{1.8\}^2 \]


\[ \mbox{Var}(X) = 3.6 - 3.24  = \boldsymbol{0.36} \]

%-----------------------------------------------------------------------------------%

%--------------------------------------------------------%
\end{document}
